\noindent 
During our Spring 2025 faculty retreat, I was assigned the task of drafting a \emph{B}ring \emph{Y}our \emph{O}wn \emph{D}evice (BYOD) policy for CS students. 

This draft policy outlines the system requirements. The earliest we can implement the policy is beginning in the academic year 2026-27, i.e, with the class of 2031.

The proposed policy here is based on similar BYOD requirements at the following colleges and universities.

\begin{itemize}
    \item \href{https://clemsonpub.cfmnetwork.com/B.aspx?BookId=11654&PageId=458060}{Clemson University}
    
    \item \href{https://www.fit.edu/policies/information-technology/policies/it-1015-bring-your-own-device-byod-policy/}{Florida Institute of Technology}
    
    \item Northern Illinois University

    \item \href{https://engineering.tamu.edu/academics/byod/index.html}{Texas A\& M (Engineering)}
\end{itemize}

Once we are satisfied with this draft, we must share it with ITS for their input. With ITS support, the proposal will need to work its way through the administration. The timing requirements are stringent, if we wish to enact the policy in AY 26-27.

When we feel that the policy will be approved, we must designed which COMP courses will require a device in the classroom. And while we may feel that all COMP courses should, we must select a few critical courses in which BYOD is absolutely essential. For example: COMP 170, 271, 272, and 363 \emph{at the very least}.

\subsection*{Checlist}
\begin{itemize}
    \item Line~\ref{line:os}: Ensure o/s and versions are current and correct
\end{itemize}


\newpage